\documentclass[12pt]{article}
\renewcommand{\linespread}{1.5}
\usepackage[utf8x]{inputenc}
\usepackage[,english,greek]{babel}
\usepackage{amsmath}
\usepackage{listings}
\usepackage{tikz}
\usetikzlibrary{arrows}
\begin{document}
	\begin{center}
    	\huge \textbf{Τεχνητή Νοημοσύνη \\ Εργασία 1} \\ $Pacman$ $PDF$\\
    	\Large Κωνσταντίνος Χαϊδεμένος\\
    	\large $sdi2200262$\\ 
	\end{center}
\vspace{36pt}
Στις υλοποιήσεις των $DFS$, $BFS$, $UCS$, $Α^*$ χρησιμοποιούμε πρώτα τον εξής κώδικα αρχικοποίησης:\\
\begin{otherlanguage}{english}
\begin{verbatim}
path = []   #create a list to store the path of the nodes        
state =  problem.getStartState()   #get the start state

frontier = ***   #data structure depending on which algorithm we are coding  
frontier.push( (state, path) )  #initialize start state in the frontier    
visited = set()    #create a set to store the visited nodes
\end{verbatim}
\end{otherlanguage}
Αυτό που θα αλλάξει για την υλοποίηση του κάθε αλγόριθμου θα είναι η δομή δεδομένων που θα χρησιμοποιήσουμε και ο χειρισμός του $while$ $loop$ για την επιλογή των $successor$ $nodes$.\\\\
Μέσα στο $while$ $loop$ το πρώτο κομμάτι του κώδικα είναι επίσης κοινό σε όλες τις υλοποιήσεις των αλγορίθμων αυτών:
\begin{otherlanguage}{english}
\begin{verbatim}
while not frontier.isEmpty():

    state, path = frontier.pop()
    #pop the top node from the frontier priority queue
    #to check if it is the goal state
        
    if problem.isGoalState(state):    #check if we have reached a goal state
            return path

    visited.add(state)              #if not add it to visited
\end{verbatim}    
\end{otherlanguage}
Έπειτα από το συγκεκριμένο κομμάτι κώδικα υπάρχει ένα $for$ $loop$ στο οποίο πραγματοποιείται ουσιαστικά η επιλογή των $successor$ $nodes$ ανάλογα με την πολιτική του κάθε αλγορίθμου. Tο συγκεκριμένο κομμάτι διαφέρει για κάθε συνάρτηση.\\\\
\section*{$Q1$ - $DFS$}
Γίνεται χρήση $Stack$ ακολουθώντας την θεωρία στις διαφάνειες του μαθήματος.\\\\
Το $for$ $loop$ είναι ως εξής:\\
\begin{otherlanguage}{english}
\begin{verbatim}
    for s in problem.getSuccessors(state):  
    #check every successor of the current state
        
        if s[0] not in visited:   
        #if the current successor is not already visited then
        
        frontier.push( (s[0], path + [s[1]]) )  
        #add the successor and the updated path to the frontier stack
\end{verbatim}
\end{otherlanguage}
\section*{$Q2$ - $BFS$}
Γίνεται χρήση $Queue$ ακολουθώντας την θεωρία στις διαφάνειες του μαθήματος.\\\\
Στη συγκεκριμένη συνάρτηση ακριβώς πρίν το $for$ $loop$ υπάρχει μια παραπάνω γραμμή κώδικα που θα συναντηθεί και στις επόμενες 2 συναρτήσεις. Ο ρόλος της είναι να μετατρέψει προσωρινά το $frontier$ από την δομή δεδομένων $Queue$ σε $list$ έτσι ώστε να γίνει εύκολα η προσπέλασή της για να γίνει ο έλεγχος μετά...\\\\
Το $for$ $loop$ μαζί με την <<έξτρα>> γραμμή κώδικα είναι ως εξής:\\\\
\begin{otherlanguage}{english}
\begin{verbatim}
   frontier_list = [ t[0] for t in frontier.list ]     
   #turn the frontier queue into a stack for checking later

    for s in problem.getSuccessors(state):              
    #check every successor of the current state
        
        if s[0] not in visited and s[0] not in frontier_list:   
        #if the successor is not already visited
        #and if the successor is not already in the frontier 
            
            frontier.push( (s[0], path + [s[1]]) )      
            # add the successor and the updated path to the frontier queue
\end{verbatim}
\end{otherlanguage}
\section*{$Q3$ - $UCS$}
Γίνεται χρήση $PriorityQueue$ ακολουθώντας την θεωρία στις διαφάνειες του μαθήματος.\\\\
Στη συγκεκριμένη συνάρτηση ρόλος της <<έξτρα>> γραμμής κώδικα είναι να μετατρέψει προσωρινά το $frontier$ από την εκάστοτε δομή δεδομένων $PriorityQueue$ σε $heap$ έτσι ώστε να γίνει εύκολα η προσπέλασή των $state$ $tuple$ για να γίνει ο έλεγχος μετά...\\\\
Το $for$ $loop$ μαζί με την <<έξτρα>> γραμμή κώδικα είναι ως εξής:\\\\
\begin{otherlanguage}{english}
\begin{verbatim}
frontier_heap = [ i[2][0] for i in frontier.heap ]     
# frontier.heap[i][2] is the state tuple: (position, path)
    
    for s in problem.getSuccessors(state):
        
        s_path = path + [s[1]]        #successor path
        
        if s[0] not in visited and s[0] not in frontier_heap:   
        #if the successor is not already visited
        #and if the successor is not already in the frontier
        
            frontier.push( (s[0], s_path), problem.getCostOfActions(s_path) )
        
        else:
            for i in range(0, len(frontier_heap)):
                if s[0] == frontier_heap[i]:
                    
                    # The stored path and the new path costs have to be compared
                    updatedCost = problem.getCostOfActions(s_path)
                    storedCost = frontier.heap[i][0]    
                    # frontier.heap[i] is a tuple: (cost, counter, (node, path))
                    
                    if storedCost > updatedCost:    
                    # we have to update the frontier heap 
                    #with the new cost and the new path
                    #however tuples are immutable so we have to do it step by step
                        
                        frontier.heap[i] = (storedCost, frontier.heap[i][1] , 
                        (s[0], s_path) )  
                        #first update the path of the successor
            
                        frontier.update( (s[0], s_path), updatedCost )   
                        #then we update the cost
\end{verbatim}
\end{otherlanguage}
\section*{$Q4$ - $A^*$}
Γίνεται χρήση $PriorityQueue$ ακολουθώντας την θεωρία στις διαφάνειες του μαθήματος.\\\\
Στη συγκεκριμένη συνάρτηση είναι λίγο <<πειραγμένο>> όλο το $while$ $loop$ γιατί για να βάλουμε νέο κόμβο στο $visted$ πρέπει να ανανεώσουμε και το κόστος του συγκεκριμένου μονοπατιού...\\\\
Το $while$ $loop$ είναι ως εξής:\\\\
\begin{otherlanguage}{english}
\begin{verbatim}
while not frontier.isEmpty():

    state, path, cost = frontier.pop()  
    #pop the top node from the frontier priority queue 
    #to check if it is the goal state

    if problem.isGoalState(state):      #check if we have reached a goal state
        return path


    if (state not in visited) or (cost < visited[state]):   
    #if the current node is not already visited 
    #or the current cost is less than the cost of the visited state

        visited[state] = cost   #update cost
            
        #for every successor calculate the cost + heuristic value and
        #add it in the frontier
        for s, s_path, s_cost in problem.getSuccessors(state):

            newpath = path + [s_path]   #calculate new path
            newCost = cost + s_cost     #calculate new cost
            newNode = (s, newpath, newCost)     
            #create a new node with new path and new cost

            frontier.push(newNode, newCost+heuristic(s, problem))
\end{verbatim}
\end{otherlanguage}
\section*{$Q5,6,7$}
Δεν πρόλαβα την λόγω πίεσης χρόνου να γράψω επεξήγηση στο $pdf$..\\\\
Ωστόσο σε όλες τις συναρτήσεις των ερωτημάτων έχω προσθέσει σχόλια σχεδόν για κάθε γραμμή οπότε μπορείτε να απαντήσετε τις ερωτήσεις σας από αυτά.
\end{document}